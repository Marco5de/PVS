\documentclass{article}
\usepackage{xcolor,listings}
\usepackage{textcomp}
\usepackage[utf8]{inputenc}
\usepackage{amsmath}
\usepackage{siunitx}
\usepackage[ngerman]{babel}
\usepackage{pgfplots}
\usepackage{hyperref}
\usepackage{pgfplotstable}
\usepackage{csvsimple}
\usepackage{float}
\usepackage[caption = false]{subfig}
\usepackage[final]{graphicx}
\lstset{upquote=true}


\begin{document}
\begin{titlepage}
    \centering
    \vfill
    \includegraphics[width=13cm]{UniUlmLogo.PNG} % also works with logo.pdf
    \vfill
    \vfill
    {\bfseries\Large
        Programmierung von Systemen\\
        Blatt 11\\
        \vskip2cm
        Marco Deuscher\\ \href{mailto:marco.deuscher@uni-ulm.de}{marco.deuscher@uni-ulm.de}\\
        Benedikt Jutz\\ \href{benedikt.jutz@uni-ulm.de}{benedikt.jutz@uni-ulm.de}\\
        \vfill
    }    
     Juni 2018
    \vfill
    \vfill
    \vfill
\end{titlepage}

\clearpage
\section{Aufgabe 2: Erreichbarkeitsanalyse}
\begin{table}[h!]
    \centering
    \vspace{3mm}
     \begin{tabular}{|c||c||c||c||c||c|}
        \hline
        Markierung & $S_1$ & $S_2$ & $S_3$ & $S_4$ & Transition \\ 
        \hline
        $M_0$ & 1 & 0 & 1 & 1 & $T_1 \rightarrow M_1$\\
        $M_1$ & 0 & 1 & 0 & 1 & $T_2 \rightarrow M_2$ or $T_4                    \rightarrow M_8$\\
        $M_2$ & 1 & 0 & 0 & 1 & $T_3 \rightarrow M_3$\\
        $M_3$ & 0 & 0 & 1 & 1 & $T_4 \rightarrow M_4$\\
        $M_4$ & 1 & 0 & 1 & 0 & $T_1 \rightarrow M_5$\\
        $M_5$ & 0 & 1 & 0 & 0 & $T_2 \rightarrow M_6$\\
        $M_6$ & 1 & 0 & 0 & 0 & $T_3 \rightarrow M_7$\\
        $M_7$ & 0 & 0 & 1 & 0 & ---\\
        $M_8$ & 1 & 1 & 0 & 0 & $T_3 \rightarrow M_9$\\
        $M_9$ & 0 & 1 & 1 & 0 & $T_2 \rightarrow M_4$\\
        \hline
        \end{tabular}
     \caption{Erreichbarkeitsanalys für gegebenes Petri-Netz}
    \label{TabelleV04}
\end{table}

Die Erreichbarkeitsanalyse zeigt, dass es wenn Zustand $M_7$ erreicht wird zu einem Deadlock kommt.

\end{document}
