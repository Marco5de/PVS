\documentclass{article}
\usepackage[utf8]{inputenc}
\usepackage{amsmath}
\usepackage{siunitx}
\usepackage[ngerman]{babel}
\usepackage{pgfplots}
\usepackage{hyperref}
\usepackage{pgfplotstable}
\usepackage{csvsimple}
\usepackage{float}
\usepackage[caption = false]{subfig}
\usepackage[final]{graphicx}


\graphicspath{ {images/} }

\hypersetup{
    colorlinks=true,
    linkcolor=blue,
    filecolor=magenta,      
    urlcolor=cyan,
}

\begin{document}
\begin{titlepage}
    \centering
    \vfill
    \includegraphics[width=13cm]{UniUlmLogo.PNG} % also works with logo.pdf
    \vfill
    \vfill
    {\bfseries\Large
        Programmierung von Systemen\\
        Blatt 8\\
        \vskip2cm
        Marco Deuscher\\ \href{mailto:marco.deuscher@uni-ulm.de}{marco.deuscher@uni-ulm.de}\\
        Benedikt Jutz\\ \href{benedikt.jutz@uni-ulm.de}{benedikt.jutz@uni-ulm.de}\\
        \vfill
    }    
     Juni 2018
    \vfill
    \vfill
    \vfill
\end{titlepage}

\section{Aufgabe: ER-Modellierung}

\begin{figure}[h]
    \centering
    \includegraphics[width = 12cm]{ER_Diagramm01.png}
    \caption{ER-Modellierung}
    \label{Versuch01_Aufbau01}
\end{figure}

In dem ER-Diagramm kann nicht dargestellt werden, dass der Skipper selbst wieder ein Crewmitglied ist. Im Oberen Diagramm ist das unsauber mit einer 'ist'-Beziehung dargestellt. Diese existiert so in einem ER-Diagramm aber nicht.
\clearpage

\section{Aufgabe: Datenbankschemata}

\begin{table}[h!]
    \centering
    \begin{tabular}{|c|c|c|c|c|}
    \hline
         \bfseries\large{Schiff} & Name & Typ & Zulassungsnummer \\
         \hline
          & & & \\
    \hline
    \end{tabular}
    \label{tabelle01}
\end{table}

\begin{table}[h!]
    \centering
    \begin{tabular}{|c|c|c|c|c|c|}
    \hline
         \bfseries\large{Crewmitglied}&Crewmitglied-ID & Name & Adresse & E-Mail & Zulassungsnummer\\
         \hline
          & & & & & \\
    \hline
    \end{tabular}
    \label{tabelle01}
\end{table}

\begin{table}[h!]
    \centering
    \begin{tabular}{|c|c|c|c|c|}
    \hline
         \bfseries\large{Wettfahrt} & Wettfahrt-ID & Name & Ort & Datum\\
         \hline
          & & & &  \\
    \hline
    \end{tabular}
    \label{tabelle01}
\end{table}


\begin{table}[h!]
    \centering
    \begin{tabular}{|c|c|c|}
    \hline
         \bfseries\large{Skipper} & Zulassungsnummer & Crewmitglied-ID\\
         \hline
          & & \\
    \hline
    \end{tabular}
    \label{tabelle01}
\end{table}


\begin{table}[h!]
    \centering
    \begin{tabular}{|c|c|c|}
    \hline
         \bfseries\large{Wettfahrtteilnehmer} & Zulassungsnummer & Wettfahrt-ID\\
         \hline
          & & \\
    \hline
    \end{tabular}
    \label{tabelle01}
\end{table}

\clearpage
%Gewonnen ergänzen
\section{Aufgabe: (min:max)-Notation}
\begin{figure}[h]
    \centering
    \includegraphics[width = 12cm]{ER_Diagramm021.png}
    \caption{ER-Modellierung}
    \label{Versuch01_Aufbau01}
\end{figure}

\section{Aufgabe: Relationenalgebra}
\paragraph{a)}
(\pi_{Teile.Bezeichnung}(\sigma_{Teile.Preis > 10€})(Teile))

\paragraph{b)} (\pi_{Stadt}(Lieferanten \cup Bestellungen))\\


\paragraph{c)} (\pi_{LiefName}(Lieferanten)) - (\pi_{KdName}(Bestellungen))

\paragraph{d)}(\pi_{Bezeichnung}(Teile \bowtie_{Teile.TeileNr = Bestellungen.TeileNr} Bestellungen))

\paragraph{e)} (\pi_{Bezeichnung}(\sigma_{Bestellungen.KdStadt = 'Berlin'\wedge Lieferaten.LiefName = 'AEG'}\\(Bestellungen \bowtie_{Bestellung.LiefNr = Lieferanten.LiefNr} Lieferanten \bowtie_{Teil.TeileNr = Bestellungen.TeileNr} Bestellungen)))










\end{document}
