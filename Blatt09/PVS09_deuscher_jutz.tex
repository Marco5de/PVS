\documentclass{article}
\usepackage{xcolor,listings}
\usepackage{textcomp}
\usepackage[utf8]{inputenc}
\usepackage{amsmath}
\usepackage{siunitx}
\usepackage[ngerman]{babel}
\usepackage{pgfplots}
\usepackage{hyperref}
\usepackage{pgfplotstable}
\usepackage{csvsimple}
\usepackage{float}
\usepackage[caption = false]{subfig}
\usepackage[final]{graphicx}
\lstset{upquote=true}


\begin{document}
\begin{titlepage}
    \centering
    \vfill
    \includegraphics[width=13cm]{UniUlmLogo.PNG} % also works with logo.pdf
    \vfill
    \vfill
    {\bfseries\Large
        Programmierung von Systemen\\
        Blatt 9\\
        \vskip2cm
        Marco Deuscher\\ \href{mailto:marco.deuscher@uni-ulm.de}{marco.deuscher@uni-ulm.de}\\
        Benedikt Jutz\\ \href{benedikt.jutz@uni-ulm.de}{benedikt.jutz@uni-ulm.de}\\
        \vfill
    }    
     Juni 2018
    \vfill
    \vfill
    \vfill
\end{titlepage}

\section{Aufgabe: SQL I}
\subsection{}
\begin{lstlisting}[
           language=SQL,
           showspaces=false,
           basicstyle=\ttfamily,
           numbers=left,
           numberstyle=\tiny,
           commentstyle=\color{gray}
        ]
SELECT *
FROM lieferanten AS l
WHERE l.LiefStadt<>"Muenchen"
\end{lstlisting}
In SQL Query wurde München korrekt geschrieben, aber Latex-Package kann Umlaute nicht anzeigen.
\hfill \break
\subsection{}
\begin{lstlisting}[
           language=SQL,
           showspaces=false,
           basicstyle=\ttfamily,
           numbers=left,
           numberstyle=\tiny,
           commentstyle=\color{gray}
        ]
SELECT DISTINCT l.TeileID
FROM liefert AS l
\end{lstlisting}

\hfill \break
\subsection{}
\begin{lstlisting}[
           language=SQL,
           showspaces=false,
           basicstyle=\ttfamily,
           numbers=left,
           numberstyle=\tiny,
           commentstyle=\color{gray}
        ]
SELECT *
FROM preisliste AS pl
WHERE pl.Preis>=300 AND pl.Preis<=430
\end{lstlisting}

\hfill \break
\subsection{}
\begin{lstlisting}[
           language=SQL,
           showspaces=false,
           basicstyle=\ttfamily,
           numbers=left,
           numberstyle=\tiny,
           commentstyle=\color{gray}
        ]
SELECT *
FROM bestellungen AS b
WHERE b.BestDatum NOT BETWEEN 2009-02-15 AND 2009-02-23
\end{lstlisting}

\hfill \break
\subsection{}
\begin{lstlisting}[
           language=SQL,
           showspaces=false,
           basicstyle=\ttfamily,
           numbers=left,
           numberstyle=\tiny,
           commentstyle=\color{gray}
        ]
SELECT *
FROM kunden AS k
WHERE k.KdStadt LIKE 'S%'
\end{lstlisting}

%TODO
\hfill \break
\subsection{}
\begin{lstlisting}[
           language=SQL,
           showspaces=false,
           basicstyle=\ttfamily,
           numbers=left,
           numberstyle=\tiny,
           commentstyle=\color{gray}
        ]
SELECT *
FROM kunden AS k
WHERE k.KdStadt LIKE 'S%'
\end{lstlisting}


\hfill \break
\subsection{}
\begin{lstlisting}[
           language=SQL,
           showspaces=false,
           basicstyle=\ttfamily,
           numbers=left,
           numberstyle=\tiny,
           commentstyle=\color{gray}
        ]
SELECT m.Name, m.Vorname, m.Gehalt
FROM mitarbeiter AS m
WHERE m.Funktion="Montage II" AND m.Wohnort="Ulm"
ORDER BY m.Gehalt ASC
\end{lstlisting}

\hfill \break
\subsection{}
\begin{lstlisting}[
           language=SQL,
           showspaces=false,
           basicstyle=\ttfamily,
           numbers=left,
           numberstyle=\tiny,
           commentstyle=\color{gray}
        ]
SELECT *,a.AuftrDatum+INTERVAL 3 MONTH  AS skontodatum
FROM auftraege AS a
\end{lstlisting}

\hfill \break
\subsection{}
\begin{lstlisting}[
           language=SQL,
           showspaces=false,
           basicstyle=\ttfamily,
           numbers=left,
           numberstyle=\tiny,
           commentstyle=\color{gray}
        ]
SELECT a.AuftrNr, a.KdNr
FROM auftraege AS a
WHERE DAY(a.AuftrDatum)%2=0
\end{lstlisting}

\hfill \break
\subsection{}
\begin{lstlisting}[
           language=SQL,
           showspaces=false,
           basicstyle=\ttfamily,
           numbers=left,
           numberstyle=\tiny,
           commentstyle=\color{gray}
        ]
SELECT *, (CASE
	    WHEN t.Bestand < t.MinBestand THEN "Mangel"
            WHEN t.Bestand > t.MinBestand THEN "Ueberbestand"
            WHEN t.Bestand = t.MinBestand THEN "Mindestbestand"
            ELSE "---"
           END) AS Bestandsbewertung
FROM teile AS t
\end{lstlisting}

\end{document}